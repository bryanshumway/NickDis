% !Tex root = dis.tex
\chapter{Implications}

Finding that the main effect was statistically significant for the Refactoring according to CLT principles but expertise was not for mean time to resolution of the bug was provocative. It suggests that fears that the Expertise Reversal Effect may make the code easier to read only for novices are as yet unsubstantiated. Finding that the perceived cognitive load of overlapping code snippets was similar for both the control and refactored case suggests that CLT may be reliable measure for the technical debt of a software project. Reliability and validity are two important attributes of useful conceptual frameworks and robust metrics, cognitive load as it relates to code may have both to offer the software engineering community. Showing that there is a statistically significant effect for refactored code that adheres to the principles of Cognitive Load Theory introduces new opportunities for the application of CLT to software engineering tooling. Taking into account the Modality Effect when developing documentation to describe a system or constructing tooling to create systems may dramatically affect the way we think about software visualization and implementation. Cognitive Load Theory itself may be informed by software engineering concepts for designing modules/interconnected systems. The principles of Information Hiding, Encapsulation, and resource independence inherent in Service Oriented Architectures may impact the way Instructional Designers organize blocks of content.
